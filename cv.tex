\documentclass[a4paper,10pt]{article}

%A Few Useful Packages
\usepackage{marvosym}
%for loading fonts
\usepackage{fontspec}
%other packages for formatting
\usepackage{xunicode,xltxtra,url,parskip}
\RequirePackage{color,graphicx}
\usepackage[usenames,dvipsnames]{xcolor}
%better formatting of the A4 page
\usepackage[big]{layaureo}
% an alternative to Layaureo can be ** \usepackage{fullpage} **
%for Grades
\usepackage{supertabular}
%custom \section
\usepackage{titlesec}

%Setup hyperref package, and colours for links
\usepackage{hyperref}
\definecolor{linkcolour}{rgb}{0,0.2,0.6}
\hypersetup{colorlinks,breaklinks,urlcolor=linkcolour, linkcolor=linkcolour}

%FONTS
\defaultfontfeatures{Mapping=tex-text}
%\setmainfont[SmallCapsFont = Fontin SmallCaps]{Fontin}
%\setmainfont{PT Sans}
\setmainfont{Ubuntu}


%CV Sections inspired by:
%http://stefano.italians.nl/archives/26
\titleformat{\section}{\Large\scshape\raggedright}{}{0em}{}[\titlerule]
\titlespacing{\section}{0pt}{3pt}{3pt}

\titleformat{\subsection}{\large\scshape\raggedright}{}{0em}{}
\titlespacing{\subsection}{0pt}{3pt}{3pt}
%Tweak a bit the top margin
%\addtolength{\voffset}{-1.3cm}


\begin{document}

% non-numbered pages
\pagestyle{empty}

%for use with \LaTeX command
\font\fb=''[cmr10]''

\par{\centering{\Huge John \textsc{Khvatov}}\bigskip\par}


\section{Personal Data}

\begin{tabular}{rl}
    \textsc{Date of Birth:} & 08 January 1989\\
    \textsc{Residence:}   & Moscow, Russia\\
    \textsc{Phone:}     & +7 925 1529926\\
    \textsc{email:}     & \href{mailto:ivaxer@gmail.com}{ivaxer@gmail.com} \\
    \textsc{LinkedIn profile:} & \href{http://www.linkedin.com/in/ivaxer}{ivaxer}
\end{tabular}


\section{Areas of interests}
\begin{itemize}
\item Scalable systems
\item Data storages
\item Modern programming languages
\end{itemize}


\section{Work Experience}
\subsection{Lead Software Developer at Under Development (undev.ru), October 2011 --- Present}
I work with team which develop platform for video processing, streaming
and storage. We use Linux, Objective-C, C, C++, Python and Go in our platform.
The most part I work at low level (C, network I/O (libev), disk I/O). The main
public projects which I've participated:
\begin{itemize}
\item CCTV of Russian presidential election in 2012 --- country-wide streaming
and storage webcasts from 200,000 webcams.
\item Internet and local video broadcasting of SPIEF 2012.
\end{itemize}
\subsection{Software Developer at Saratov State University, PRCNIT, July 2008 ---
October 2011}
PRCNIT (Povolzhsky Regional Centre of New Information Technologies) is a division
of Saratov State University, which operates in the IT-sphere. I worked in the
network and telecomunication systems department, which mainly deals with
the developing of University infrastructure and testing of new technologies. Among
my responsibilities were:
\begin{itemize}
\item VoIP telephony development. I have worked on VoIP telephony design and
development for the University. When I joined the University the telephony included
three analog PBXes (three buildings were covered) connected via Asterisks (IAX2),
one PBX was connected to PSTN gateway via E1. Now the telephony system is based on
SIP protocol with OpenSIPS as registrars and proxis, FreeRADIUS as OpenSIPS
backend for AAA and Asterisk as gateways to legacy analog PBXes. The telephony
system scales for the University requirements, it is easy manageable, fault tolerant.
\item Infrastructure management software creation. Here are some pieces of
software that I've written:
 \begin{itemize}
 \item Column based storage in Scala for storing netflow data
(\href{http://github.com/ivaxer/cbs}{source code}). The project is still at an
early stage but first results show decrease in disk space usage by 30\%. It's
planned 50-60\% after the all features are implemented.
 \item virt-platform --- lightweight platform for creating private clouds
(\href{http://git.sgu.ru/?p=virt-platform.git;a=summary}{source code}).
The platform is based on libvirt. It's designed to extend libvirt
while maintaining its flexibility, support several storage backends (it currently
supports only DRBD, Ceph\slash RBD is planned).
 \item Software for provisioning Linksys SPA SIP-phones
(\href{http://git.sgu.ru/?p=spaconf.git;a=summary}{source code}). Spaconf is
implemented as a WSGI application that receives phone's
parameters from an internal database, generates phone config as described in
specification from Linksys, crypts and sends config back to the phone.  Spaconf
was deployed to production two years ago.
 \item Module for FreeRADIUS in Python, which works as AAA backend for HP
procurve switches (\href{http://git.sgu.ru/?p=aaa\_hp.git;a=summary}{source
code}).
 \item Dozens scripts in Python for automation, data processing and monitoring.
 \end{itemize}
\item Testing and introducing new technologies and infrastructure. Open source
technologies that I implemented included RPM build system (extended Plague to
support the packages building directly from GIT repository), GIT hosting
(gitosis), Hudson(Jenkins) CI.
\item Refactoring, improvement of performance, localizing memory leaks in open
source software that we use (OpenSIPS, Asterisk, rdiff-backup and PRCNIT's
software).
\end{itemize}

\subsection{VoIP-platform Developer at TipMeet, April 2010 --- Present}
TipMeet is a startup that provides paid lines. I develop
a communication platform. The platform is built of the following components:
OpenSIPS, MediaProxy, FreeSWITCH, SIP B2BUA based on Sippy and
tippresence.\\
At TipMeet I have:
\begin{itemize}
\item performed a part of work on the design of the communication platform.
\item developed a presence server
(\href{http://github.com/tipmeet/tippresence}{source code}) with SIP\slash
HTTP\slash AMQP interfaces on Twisted\slash Python. As a part of the project
I've developed a simple SIP-stack (incomplete implementation of RFC3261)
(\href{http://github.com/tipmeet/tipsip}{source code}) on
Twisted. The presence server is used in production now.
\item developed routing logic for OpenSIPS and internal RESTful API
to connect OpenSIPS to the project database. The RESTful API is implemented as
a WSGI application.
\item assisted with bugs fixing in the complex multicomponent system.
\end{itemize}


\section{Education}
\begin{tabular}{rl}
2006 --- 2011& M.S. in \textsc{Computer Science} and \textsc{Information
Technologies} \\& \textbf{Saratov State University}
\end{tabular}


\section{Related activity\slash industry participation}
\begin{itemize}
\item OpenSIPS community member, source code contributor and maintainer in
Fedora/RHEL repositories.
\item Google Summer of Code 2010 program participant within Asterisk project.
\item Member of Fedora project.
\item Developer of software package for modeling the atomic structures on the
computer clusters using MPI technology.
\end{itemize}

\end{document}
